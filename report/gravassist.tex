\documentclass{article}
\usepackage{tikz}
\usepackage{braket}
\usetikzlibrary{datavisualization}
\usetikzlibrary{calendar}
\usetikzlibrary{graphs}
\usetikzlibrary{positioning}
\usetikzlibrary{arrows.meta,decorations.pathmorphing,backgrounds,positioning,fit,petri}
\usetikzlibrary{mindmap,shadows}
\usetikzlibrary{angles,calc,quotes}
\usetikzlibrary{positioning}
\usetikzlibrary{arrows.meta}
\usetikzlibrary{patterns,patterns.meta}
\usetikzlibrary{shapes.geometric}
\usepackage{setspace}
\usepackage{multicol}
\usepackage{pst-node}
\usepackage{mathtools}
\usepackage{wrapfig}
\usepackage{pgfplots}
\usepackage{multirow}
\usepackage{siunitx}
\usepackage{graphicx}
\usepackage{csquotes}
\usepackage{amsmath, amssymb}
\usepackage{amsfonts, amsthm}
\usepackage{empheq}
\usepackage{float}
\usepackage[spanish]{babel}
\usepackage{booktabs}
\usepackage{subcaption}
\usepackage{hyperref}
\usepackage[backend=biber,style=apa]{biblatex}

\addbibresource{bib.bib}
\usepackage[a4paper]{geometry}
\geometry{top=1in, bottom=1in, left=1in, right=1in}
\setlength{\parindent}{0.5in}
\renewcommand{\baselinestretch}{2}
\pgfplotsset{compat=1.18}

\newcommand{\titulo}{}
\def\titulo(#1,#2){
\vspace*{6cm}
\pagestyle{empty}
\begin{center}
	{\huge \textbf{#1}}

    \vspace*{.5cm}

	{\LARGE#2}
\end{center}
\newpage
\tableofcontents
\newpage
\vspace*{.5cm}
\begin{center}
	{\LARGE \textbf{#1}}
\end{center}
\vspace*{1cm}
}
\counterwithin*{section}{part}
\begin{document}

\titulo(Análisis de maniobra de asistencia gravitacional:\\ ``slingshot'', Mecánica Clásica I \\ Eneko Acevedo Pérez)
\vspace{3cm}
\begin{figure}[ht!]
	\centering
	\begin{tikzpicture}
		\filldraw[yellow] (0,0) node[black]{Sol} circle (.5);
		% órbitas
		\draw[] (0,0) circle (6);
		\draw[] (0,0) circle (4);
		\draw[red, thick] (6,0) arc(0:90:5 cm);
		% \draw[thick, red] (1,0) ellipse (5 and 4.5);
		% planetas
		\filldraw[blue!50] (6,0) node[black,below left]{Tierra} circle (.2);
		\filldraw[orange!50] (2.82,2.82) node[black,above right]{Venus} circle (.2);
		\node[dart, draw, fill, red, rotate=180] at (1.1,5) {};
	\end{tikzpicture}
\end{figure}

% ------------------------------------------
\newpage
% ------------------------------------------
\pagestyle{plain}
\section{Introducción}

El presente trabajo aborda la simulación de un sistema solar reducido compuesto por el Sol, la Tierra, Venus y una nave espacial.

% La metodología implementada se basa en la integración numérica de las ecuaciones de Newton utilizando el método de Runge-Kutta de cuarto orden (\cite{rk4}). Debido a las variaciones en la magnitud de la fuerza gravitatoria durante los encuentros cercanos planetarios, se ha desarrollado un algoritmo de paso de tiempo adaptativo. Este refinamiento permite mantener la estabilidad numérica y la conservación de la energía mecánica del sistema cuando la nave atraviesa la esfera de influencia de Venus, ajustando la resolución temporal en función de la distancia relativa entre los cuerpos.

El objetivo central del estudio es el análisis y optimización de la maniobra de asistencia gravitacional. Se examina el intercambio de momento angular entre el planeta y la nave para determinar las condiciones iniciales de lanzamiento y el parámetro de impacto que maximizan la velocidad heliocéntrica final de la sonda. Se construye una superficie de optimización, mutando las variables que afectan al intercambio de energía, que permite identificar la trayectoria de máxima ganancia.

\subsection{Plan de vuelo}
Se va a proceder con dos rutinas de cálculo numérico para la búsqueda de la ruta 
óptima, para luego comparar sus resultados.

Como primera opción tenemos el método analítico. El Sol se considerará
fijo en toda la trayectoria de estudio. Al principio de la órbita de transferencia,
solo se considerará la acción del Sol sobre la nave para obtener una elíptica dependiente
de la velocidad inicial. Con esto se tendrán dos puntos de corte con la órbita
de Venus para cada trayectoria. Para 
cada punto de corte, se modelará el sistema Venus nave variando la posición de Venus
en un entorno centrado en el corte. El punto exacto en que se cambia de sistema 
de referencia queda determinado por el radio de
la esfera de influencia gravitatoria de Venus
(\cite{rsoi}).

La segunda opción es completamente simulada a ordenador. Se tendrán en cuenta 
la Tierra, Venus, el Sol y la nave en todo momento y el paso de tiempo será adaptativo
para salvar singularidades y mejorar la precisión. Este paso de tiempo será dependiente 
de la distancia de la nave al cuerpo más cercano.

% ------------------------------------------
\newpage
% ------------------------------------------
\part{Cálculo Analítico}

En esta parte del trabajo abordamos el problema dividiendo la trayectoria en dos regímenes dinámicos diferenciados, una técnica conocida en astrodinámica como \textit{Patched Conics}. Para cada régimen, simplificamos el sistema general de $N$ cuerpos reduciéndolo a problemas locales de dos cuerpos, lo cual nos permite aplicar soluciones analíticas exactas.

\section{La transferencia heliocéntrica}

Para la fase de crucero entre la Tierra y Venus, consideramos el sistema Sol-nave aislado. Según la teoría de fuerzas centrales, este sistema de dos cuerpos puede reducirse a un problema efectivo de un solo cuerpo de masa reducida $\mu$ moviéndose en un potencial central $V(r)$ (\cite{teoria_mecanica}, p. 44).

Dado que la masa de la nave es despreciable frente a la del Sol ($m \ll M_{\odot}$), la masa reducida $\mu \approx m$ y el centro de masas coincide prácticamente con el Sol. La dinámica queda gobernada por la conservación de la energía mecánica $E$ y del momento angular $\vec{L}$, lo que confina la trayectoria a un plano invariante.

La ecuación de la \textit{vis-viva}, derivada directamente de la integral de la energía para el potencial gravitatorio newtoniano, determina la velocidad en cualquier punto de la elipse de transferencia:

\begin{equation}
    v^2 = \mu \left( \frac{2}{r} - \frac{1}{a} \right)
\end{equation}
donde $v$ es la velocidad orbital de la nave, $r$ la distancia heliocéntrica, $a$ el semieje mayor de la elipse (definido por la energía total negativa del sistema, $E < 0$) y $\mu=GM$ es el parámetro gravitacional estándar del cuerpo al que se orbita.

% Geométricamente, determinamos los puntos de corte imponiendo la condición $r(\theta)_{nave} = r(\theta)_{v}$. Sin embargo, la viabilidad de la maniobra de asistencia gravitacional depende críticamente de la fase orbital relativa en el momento del encuentro.

\section{Dispersión en la Esfera de Influencia (SOI)}

Cuando la nave ingresa en la Esfera de Influencia de Venus ($r < r_{\text{SOI}}$), realizamos un cambio de sistema de referencia. El problema se modela ahora como la interacción de una partícula libre con un centro dispersor fijo (Venus). Definimos el límite de esta interacción mediante la aproximación de Laplace (\cite{rsoi}):

\begin{equation}
    r_{\text{SOI}} \approx r_{\text{v}} \left( \frac{m_{\text{v}}}{M_{\odot}} \right)^{2/5}
\end{equation}
donde $r_{\text{v}}$ es la distancia del Sol a Venus, $M_\odot$ es la masa del 
Sol y $m_{\text{v}}$ la masa de Venus.

En este marco local, la nave posee una energía mecánica positiva respecto a Venus ($E > 0$), lo que resulta en una trayectoria abierta hiperbólica. El objetivo del \textit{slingshot} se reinterpreta como un problema de dispersión: buscamos maximizar la deflexión del vector velocidad asintótico.

La geometría del encuentro viene caracterizada por el parámetro de impacto $s$ (distancia perpendicular asintótica al centro de fuerzas). Según la teoría de dispersión en campos centrales conservativos, el ángulo de dispersión $\theta$ (el ángulo total que gira la velocidad de la nave) está relacionado unívocamente con $s$ mediante la expresión (\cite{teoria_mecanica}):

\begin{equation}
    s = \frac{k}{2E} \cot \left( \frac{\theta}{2} \right)
	\quad\implies\quad
	\theta = 2 \tan^{-1} \left( \frac{k}{2Es} \right)
\end{equation}
donde $k = G M_{\text{v}} m$ es la constante del potencial gravitatorio y $E = \frac{1}{2} m v_{\infty}^2$ es la energía cinética incidente, que será aproximada a la velocidad relativa con respecto a Venus cuando la nave entre en su esfera de influiencia. Esta relación analítica nos permite apuntar con precisión: variando ligeramente $s$ en nuestra simulación, controlamos directamente el ángulo de salida y, por tanto, la ganancia de momento lineal en el sistema heliocéntrico.



% ------------------------------------------
\newpage
% ------------------------------------------
\part{Simulación Numérica de 4 Cuerpos}

En esta parte, en lugar de ``apagar'' y ``encender'' la gravedad de los distintos cuerpos, simulamos la evolución simultánea de todo el sistema (Sol, Tierra, Venus y nave) integrando directamente las ecuaciones del movimiento. Esto permite capturar las perturbaciones gravitatorias continuas que el método analítico ignora.

\section{Lagrangiano del sistema}

Para un sistema de $N$ partículas interactuando gravitatoriamente, el estado dinámico queda definido por el Lagrangiano $L = T - V$ (\cite{teoria_mecanica}), donde $T$ es la energía cinética total y $V$ la energía potencial del sistema:

\begin{equation}
    L = \sum_{i=1}^N \frac{1}{2} m_i |\dot{\vec{r}}_i|^2 + \sum_{i<j} \frac{G m_i m_j}{|\vec{r}_i - \vec{r}_j|}
\end{equation}

Aplicando las ecuaciones de Lagrange $\frac{d}{dt}\frac{\partial L}{\partial\dot{q}_{j}}-\frac{\partial L}{\partial q_{j}}=0$ a cada coordenada de la nave, obtenemos la ecuación diferencial vectorial acoplada que gobierna su aceleración:

\begin{equation}
    \ddot{\vec{r}}_{nave} = - \sum_{j \neq nave} G m_j \frac{\vec{r}_{nave} - \vec{r}_j}{|\vec{r}_{nave} - \vec{r}_j|^3}
\end{equation}

Esta ecuación no tiene solución analítica cerrada para $N > 2$, por lo que recurrimos a la integración numérica mediante el método de Runge-Kutta de cuarto orden (RK4).

\section{Integración temporal y paso adaptativo}

Hay que tener en cuenta que el término de energía potencial efectiva diverge cuando la distancia relativa tiende a cero ($V \propto -1/r$).

Durante el \textit{slingshot}, la nave pasa muy cerca de Venus (en el periapsis), donde la curvatura de la trayectoria y la magnitud de la aceleración cambian drásticamente. Un paso de tiempo fijo ($\Delta t$) sería computacionalmente ineficiente en el espacio interplanetario (demasiado pequeño) e impreciso en el encuentro cercano (demasiado grande).

Para solucionar esto, se implementa un algoritmo de paso de tiempo adaptativo. La simulación ajusta dinámicamente $\Delta t$ en función de la distancia al cuerpo más cercano ($r_\text{min}$):

\begin{equation}
    \Delta t = \eta \cdot r_\text{min}
\end{equation}
donde $\eta$ es un factor de escala empírico. Esto asegura que la resolución temporal aumente automáticamente en las regiones donde el gradiente del potencial es mayor.

\section{Validación mediante leyes de conservación}

Dado que el Lagrangiano del sistema no depende explícitamente del tiempo ($\partial L / \partial t = 0$), la energía mecánica total (Hamiltoniano) debe conservarse (\cite{teoria_mecanica}):

\begin{equation}
    H = \sum_{i} \frac{|\vec{p}_i|^2}{2m_i} + V(\vec{r}_1, ..., \vec{r}_N) = \text{cte}
\end{equation}

Utilizamos esta propiedad teórica como herramienta de control de calidad. Monitorizando la variación de la energía total del sistema durante la simulación, verificamos que el error numérico introducido por el integrador RK4 se mantiene acotado, validando así la precisión física de la trayectoria de \textit{slingshot} obtenida.


% ------------------------------------------
\newpage
% ------------------------------------------
\phantomsection
\addcontentsline{toc}{part}{Discusión de Resultados y Conclusiones}
\part*{Análisis de Resultados y Consluciones}


% ------------------------------------------
\newpage
% ------------------------------------------
\printbibliography
\end{document}

